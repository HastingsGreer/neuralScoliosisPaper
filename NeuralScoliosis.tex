\documentclass{llncs}
%
\usepackage{makeidx}
\usepackage{graphicx}
\usepackage{mwe}
\usepackage{subfig}
\usepackage{pgfplotstable}
\usepackage{pgfplots}
%
\begin{document}
%
\frontmatter
%
\pagestyle{headings}
\addtocmark{Neural Scoliosis}
\title{Scoliosis Screening using Self Contained Ultrasound and Neural Networks}
\titlerunning{Scoliosis Screening}
\author{Hastings Greer \and Stephen Aylward \and Sam Gerber \and Matt McCormick \and Deepak Chittajallu \and Neal Siekierski }
\institute{Kitware Inc, Carrboro NC 27510, USA}
%
\mainmatter
\maketitle

\begin{abstract}
We aim to diagnose scoliosis using a self contained ultrasound device that will not require expert operation. The device will detect the angle between the device and the spine using a neural network, and measure the angle between the device and vertical using an off the shelf IMU. The difference between these values will produce a plot of the spine deviation from vertical during the scan, and the extrema of this plot will yield the Cobb angle.
\end{abstract}


\section{Motivation}
Currently scoliosis diagnosis and tracking requires either X-Ray imaging, which, while very accurate, involves radiation and is impractical for in school screening, or visual inspection, which requires training, is subjective, and requires an X-ray for confirmation. We hope to replace both of these using a hand-held ultrasound wand.


\section{Training Data}
Our initial study was conducted on 




\section*{Acknowledgments}
This work was funded, in part, by the following grants.
\begin{itemize}
	\item NIH/NIBIB: In-field FAST procedure support and automation (R43EB016621) 
	\item NIH/NIGMS/NIBIB: Slicer+PLUS: Collaborative, open-source software for ultrasound analysis (R01EB021396)
\end{itemize}

%
% ---- Bibliography ----
%

\bibliographystyle{plain}
\bibliography{references}



\end{document}

